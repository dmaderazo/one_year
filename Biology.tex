% Gather round children, we are going to tell the story about protein synthesis!

% A long time ago, before the world was in colour, biologists were having lots of fun doing purifying assays and looking for proteins, carbohydrates, lipids and all sorts of other fun biomolecules. In 1868, some dude who's as old as I am now stumbled upon something that wasn't any of the things I just listed and promptly became immortalized as the guy who discovered Deoxyribonucleic Acid (DNA). (As a side note, he called it something else back then.)

% Now, DNA is described as being a linear molecule composed of two anti-parallel strands that are twisted together. (Imagine a really long rope ladder fixed at the top with some kid who liked spinning on the bottom). The "sides" of the ladder are made up of a sugar phosphate  backbone and the "rungs" of the ladder are made up of pairs smaller molecules known as nucleotide bases: adenine (A), cytosine (C), guanine (G) and thymine (T). The rungs of the ladder are pairs made between AT and CG molecules (also known as base pairs). Back in 1953 Watson and Crick stole the structure of DNA from Franklin. There's some other words that we have to know like histone (basically storage) and chromosome (basically storage). 

% Here's how we start from DNA and get proteins out of it:

% DNA is typically stored in a very tightly wound configuration stored in the nucleus of cells. When protein synthesis is initiated, DNA uncoils and makes the sequence accessible.In a process known as transcription, a combination of proteins known as transcription factors (TFs) and the enzyme RNA polymerase interact with DNA creating a single stranded molecule known as Ribonucleic Acid (RNA). Another distinction between RNA and DNA is that the sugar backbone is composed of a ribose sugar and the nucleotide base thymine is replaced with uracil (U) in RNA. Once RNA is created, it exits the nucleus 

Deoxyribonucleic Acid (DNA) is a linear, dual stranded molecule comprising of a sugar phosphate backbone. Each strand of DNA is made up of smaller molecules called nucleotide bases: adenine (A), cytosine (C), guanine (G) and thymine (T). The structure of the molecule can be though of as a dual helix or twisted ladder where the sugar phosphate backbone is the side rail. In this analogy, the rungs of the ladder are matched pairs of AT or CG molecules, known as base pairs. DNA can be thought of as the "reference" of genetic material in the organism as it contains all the information necessary for building and maintaining the organism. This information is stored in the sequence of nucleotide bases and sections of DNA that are used in the production of protein synthesis are known as genes. 

Before describing protein synthesis, Ribonucleic Acid (RNA) needs to also be introduced. Unlike DNA, RNA is a typically single stranded molecule with a ribose sugar backbone. RNA is also made up of the same nucleotide bases as DNA with the exception of thymine being replaced by uracil (U). If DNA is thought of as the reference, RNA is akin to some sort of copy that is used for short term storage. 

In protein synthesis, proteins called transcription factors bind to regulatory regions in the DNA known as enhancer and promoter regions to recruit an enzyme known as RNA Polymerase. This enzyme is responsible for the production of RNA. Once the transcription is completed, RNA leave the nucleus where DNA is stored. The RNA is then processed in a separate part of the cell, the ribosome, as the teamplate to create protein. 
The process of protein synthesis can be summarised in what is commonly called "The Central Dogma of Biology" which states that DNA gets transcribed RNA which gets translated into protein. 

REASONS 

{\color{red} Gotta talk about the other types of RNA. What do they do? Coding/non-Coding? TFBS? How much more detail?}