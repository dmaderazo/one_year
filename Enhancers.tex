\section{Enhancers}
    Transcription factors are important because they modify the transcriptional machinery
    
    How do they do it?
        Gene expression is crucial to the function and differentiation of different cell types and required for growth, repair and maintenance of an organism.  In eukaryotic cells, genes are embedded in DNA and collected in tightly packed configurations, known as chromatin, in the nucleus of cells~\cite{alberts2002chromosomal}.  A gene is a region in the DNA that contains the necessary information for the synthesis of an associated protein.
        The types of genes and the intensity associated with their expression in a cell is known as an expression profile. All the nucleated cells in an organism contain the same DNA, but the differences in their function arises from the different expression profiles~\cite{lockhart2000genomics}. Our interest lies in the identification and classification of certain regions in the DNA that help regulate the expression of particular genes. 
       

        One of the enduring tenets of molecular biology is that gene expression occurs as a result of DNA being transcribed to RNA and then translated to proteins~\cite{crick1958protein}.During the first stage, proteins called transcription factors (TFs) bind to regulatory regions surrounding the gene to recruit RNA polymerase; an enzyme responsible for the synthesis of a sub type of RNA known as messenger RNA (mRNA). 
        The second phase, known as translation, is primarily carried out by a molecule called a ribosome and by translational RNA (tRNA) outside of the nucleus. The end result is a protein comprised of an amino acid chain.
        
        Gene expression can be regulated in a variety of stages, but we focus on regulation in the context of transcription. 
At the transcriptional level, regulation requires involvement of a large number of factors, proteins and enzymes with a variety
 of roles~\cite{lemon2000orchestrated}. 
The main mechanism by which TFs interact with DNA, and subsequently regulate transcription
. the In this work, we focus on transcriptional regulation, through enhancers.

TFs are able to recognize short sequences in non-coding regions of DNA and bind to these sites. 
Upon binding, TFs recruit other TFs to bind to other sites on the DNA or to the proteins themselves.
The exact mechanism of how these TF complexes work is not understood very well. 
One plausible explanation is that some combination of TFs need to be succesfuly bound to DNA to carry out their function.
Another theory suggests that TFs can bind in a nonb-specific way to DNA sequences as well as other TFs, introducing 
a dimension of protein-protein interaction in to the activation of the reglatory element. 
There are other suggestions on how TFs might bind cooperatively in order to become active regulatory elements; for a review, 
see \textbf{Spitz and Furlong 2012 CITE}. 


Here are the main ways that gene expression might be altered:
\begin{itemize}
	\item Direct changes to DNA sequence. This is known as mutation and is typically "observed" as an organism evolves
		For example, we can see in relatively closely related species D.Melangocaster and D.Simulans, that the
		gene EXAMPLE GENE has mutated resulting in either a different expression pattern or a different gene.
		In extreme cases, this mutation may result in a different gene or even a silenced gene that is no longer
		expressed.
	\item Epigenetic changes. Gene expression is a complex and multifaceted process. Regulation in the expression
		of a gene happens at different levels, at different stages, by different factors. For the purposes
		of this work, a focus will be placed on regulation at the transcriptional level. Specifically, the focuse
		will be placed on a subset of TFs. We would like to identify and classify these enhancers at a stage
		at a sequence level.
\end{itemize}
        
TF binding sites (TFBS) are divided into categories according to the function associated with the TFs that bind to that site.
We would like to focus on the category known as Enhancers. Enhancers are a type of TFBS that enhance or upregulate the 
expression of a gene associated with that site.
To compount things a single enhancer may act on multiple genes at some distance away from the gene of interest. 


        Changes in gene expression can often be attributed to changes in either the direct genetic sequence (\emph{eg.} mutation), or epigenetic factors in the expression process. There are numerous ways that epigenetic factors can alter the expression of genes without changes to the genetic code itself. A readily available example of an epigenetic factor are TFs. These TFs take on a diverse set of roles in regulating transcription, such as the initiation of the process and the recruitment of other factors~\cite{lemon2000orchestrated}.
        
        
        Conservation of sequence as a single metric has been used for the identification of enhancers. This approach has proven useful in the past, but in the context of cardiac enhancers, relatively few have been found. Sequencing has been used, in conjunction with knowledge about binding proteins~\cite{blow2010chip}
        
        Enhancers may show conservation of function across a large evolutionary distance, even if the sequences have diverged significantly~\cite{tautz2000evolution}
        
        That doesn't mean that there is no loss of sequence conservation. It has been shown in tissue specific studies that gene expression is much higher conserved than TF binding. Positive correlation with combinatorial TF binding and transctipion levels of closeby genes~\cite{wong2014decoupling}
        
        We have some data in the form of ChIP-Seq h3K4me1. This denotes the methylation status of a histone. This particular site often associated with promoter regions of actively being transcribed genes~\cite{barski2007high}. 
        %https://epigenie.com/key-epigenetic-players/histone-proteins-and-modifications/histone-h3k4/
        While this association has been used to identify active and primed enhancers, it is poorly understood whether H3K4me1 influences, is influenced by or correlates enhancer activity~\cite{rada2018h3k4me1}
        
        The many faces of histone lysine methylation 2002)
        
        Revisit Spitz and Furlong to see what sort of idease we have to give an overlay of what a TF is and how we have different models for regulation. 
        
       
        
        
        
        As an example, the remodelling of chromatin through histone methylation can alter the accessibility of transcriptional machinery to reach their target genes~\cite{gibney2010epigenetics, holoch2015rna }. 
        
        % Changes in gene expression is primarily driven by direct changes in the sequence (genetic) or changes in epigenetic factors involved in the expression process, such as chrmomatin remodelling through histone methylation altering accessibility between transcriptional machinery and their target gene~\cite{gibney2010epigenetics, holoch2015rna }.% (RNA CAN ALSO REGULATE GENE EXPRESSION MAYBE) 
        Epigenetics is important~\cite{holliday2006epigenetics}
        DNA Methylation is super important when considering stability of gene epxression~\cite{jaenisch2003epigenetic}
        
        Comparative genomics might not be the best way since these things display modest or no conservation. Sequence conservation does not provide clues to function, even if it is identified as an enhancer~\cite{pennacchio2013enhancers}
        
        
        We have many different 
