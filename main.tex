\documentclass[12pt,a4paper]{report}
\author{D Maderazo}
\usepackage{amsmath, amsthm, amssymb, textcomp, enumerate, multicol, fancyhdr, varwidth, graphicx, color, mathrsfs}
\usepackage[left=25mm,right=25mm,top=30mm,bottom=30mm]{geometry}
\usepackage{amsfonts}
\usepackage[]{algorithm2e}
\usepackage{csquotes}
\linespread{1}
\newtheorem*{mydef}{Definition}
\newtheorem{mythm}{Theorem}
\pagestyle{fancy}
\lhead{}
\rhead{Dominic Maderazo}
\newcommand*\dee{\mathop{}\!\mathrm{d}}
\newcommand*\del{\mathop{}\!\partial}
\def\*#1{\mathbf{#1}}

        
\begin{document}

\title{Application of Bayesian Techniques for Genome Segmentation Analysis}
\date{\today}
\author{Dominic Maderazo\and Supervisors: Jennifer Flegg \& Jonathan Keith}
%        \and Richard Row, \LaTeX\ Academy}
\maketitle

%\tableofcontents
\chapter{Literature Review}

    \section{Protein Synthesis}
{\color{red} CITATIONS PLEASE}

%This section aims to provide a basic introduction to the biology of protein synthesis. 
%DNA is a long macromolecule consisting of a sugar phosphate backbone with attached with nucleotide bases Guanine, Cytosine, Adenine and Thymine (abbreviated as G, C, A,  and T) (CITATION)

%Yo, so DNA is like a double stranded, long stringy boy in a double helix thing with some nucleotide stuff in the middle. It's long, like really long. Proportionally longer than Jono long but that's okay because it's really tightly wound. Instead of having one super long DNA string with everything in it, the genome is divided up into these chromosome bois and they're conveniently stored in the nucleus (in cells that have them).

%eople were kinda unsure about how DNA looked for ages, but then some white dudes did some Xray chrystallography and some math and argued with some other white dudes and we ended up with the molecular structure that we have now. 

%In the DNA there are these bits that are kinda important because they help make proteins and proteins are kinda important. These important bits are called genes. All cells hold the same genome in a given organism but they all express different genes so that the different cell types can do different stuff. There are some caveats here about the fact that even though there's alike a bazillion nucleotides in the DNA of an organism, only a teensy bit isn't JUNK. 

%To make a protein, some squishy lads (TF \& RNA Polymerase) come along and bind near the transcription start site and get going. This makes some RNA. RNA is like DNA, except halved so it's got only one backbone (that's made of NOT sugar-phosphate) and one of the nucleotides is a U instead of a T. 

%Then the RNAs get out of the nucleus and they go to 
%LOL JOKE, turns out most of the junk bits actually help with the initiation and regulation of this fancy transcription process. There are heaps of tricky models for how enhaceosomes wotk but the main point is that they're out there.

The mechanisms underlying the transfer of genetic information was of great interest to scientists in the 20th century.
Experiments performed by Avery, McCleod and McCarty in 1944 demonstrated the molecule deoxyribonucleic acid (DNA) as the carrier for genetic information~\cite{macleod1944studies}. While this had provided evidence for the existence of some molecule of heredity, it was not until Watson and Crick~\cite{watson1953structure} performed X-ray crystallography experiments and suggested the now familiar structure for DNA that provides the basis for heredity mechanics. DNA is a long molecule consisting of a sugar phosphate backbone chains arranged in a double helix held together by pairs of nucleotide bases: cystine (C), guanine (G), adenine (A) and thymine (T). It was Crick~\cite{crick1958protein} that proposed one of the most enduring ideas of molecular biology; the \emph{Central Dogma}. Simply, the Central Dogma says that DNA is transcribed to make ribonucleic acid (RNA); which are then translated to proteins. 
The structure of DNA is the pivotal underlying idea behind the central dogma.


Proteins play an important since they are required for the repair, maintenance and growth of an organism. 
A gene is a region in the DNA that contains the necessary information for the synthesis of an associated protein. In eukaryotes, genes are embedded in DNA and collected in tightly packed configurations in the nucleus of cells ~\cite{alberts2002chromosomal}. Typically, genes are flanked by regulatory regions that act as binding sites for enzymes to assist in the initiation and regulation of gene expression.
All the nucleated cells in an organism contain the same DNA but different specialized cell types each each express a different profile of genes owing to their different roles and function~\cite{lockhart2000genomics}. 
%A protein complex known as chromatin helps in the storage of DNA by wrapping 

%The process of transcribing deoxyribonucleic Acid (DNA) to ribonucleic Acid (RNA) and the subsequent translation from RNA to protein is fundamental to the study of molecular biology. So much so that this process, first suggested by Crick, is often widely known as the central dogma of molecular biology \cite{crick1958protein}. Below is a brief summary of the relevant molecules, proteins and enzymes involved in the process. 

According to the central dogma, protein synthesis occurs in two stages; the first being the transcription of DNA to RNA. 
During the first stage, proteins called transcription factors bind to regulatory regions surrounding the gene to recruit RNA polymerase; an enzyme responsible for the synthesis of a sub type of RNA known as messenger RNA (mRNA). Note that these regulatory regions are not transcribed in this process. 
The mRNA is able to leave the nucleus for further processing, due to the smaller size of the molecule. 

However, not all of the sequence encoded in the DNA, and subsequently RNA, is required for protein synthesis. Translated regions are further subdivided in to introns and exons. Splicing is a process in which enzymes cut the introns out of the gene sequence and join the exons (see fig~\ref{fig:splicing}).

    \begin{figure}
        \centering
        {\color{red} \textbf{insert diagram of splicing}}
        %\includegraphics{}
        \caption{The processing of transcripted sequence removing the intronic regions, making the final mRNA sequence}
        \label{fig:splicing}
    \end{figure}
    

    

The second phase, known as translation, is primarily carried out by a molecule called a ribosome and by translational RNA (tRNA) outside of the nuclues. These tRNAs are another sub-type of RNA that can bind to amino acids. The ribosomes bind to the spliced mRNA sequences and are responsible for recruiting tRNAs, {\color{red}unloading} their amino acids and forming amino acid chains. The complete chain of amino acids is (in essence) the {\color{red}final} product, known as a protein. 

\subsection{Non-coding things}
There are some competing schools of thought regarding the changes of gene expression; the main two being genetics and epigenetics. While both fields are concerned with the study of changes in gene function and heredity, the main motivating ideas behind both are fundamentally different. Genetics is concerned with the change of expression brought about through changes in the genetic sequence. In contrast, epigenetics is concerned with the change of gene expression and function by external factors from the actual genetic material itself, such as with histone modification. The work in this thesis will focus on things from a genetics perspective. As such, a common way to change the expression of genes is through the change of surrounding regulatory regions.

These enhancers are regions flanking the genes and contain binding sites for proteins (transcription factor) that regulate the expression of corresponding genes. There exist different models for the ways that these enhancers work for the TFs; a complete review may be found in {\color{red} CITATION}. In summary, there are like three and they all behave slightly differently for protein-protein dynamics and stuff.

\subsection{Epigenetics}
    WHAT IS A THING
        \begin{itemize}
            \item Changes in the environment can change phenotype. 
        \end{itemize}
\subsection{Enhancers}
	DNA structure consists of four nucleic acids: Guanine, Cytosine, Adenine and Thymine (abbreviated as G, C, A,  and T) attached to a sugar phosphate back bone.
	The {\color{red}entirety of the genetic material (What about E P I G E N E T I C S)} in eukaryotic organism is coded as a long sequence of these nucleotides. Initial estimates for the protein coding proportion of the genome was placed at  2\%~\cite{international2001initial}. 
	The remainder of the non-coding regions were originally though to be \emph{junk} DNA. This thinking has progressively changed over time and non-coding regions have been shown to be locations for transcription factor binding sites for the assistance in the regulation of gene expression. {\color{red} The ENCODE project estimates that 80\% of the genome is functional, far greater than $<$2\% coding for proteins \textbf{citation}}
	It has been theorized that the conservation of these non-coding regions throughout evolutionary distances is some indication of function.
	
	Regions surrounding important genes such as the SOX2 gene have been shown to show a local neighbourhood lacking in other genes but heavily populated by conserved enhancer regions accross species {~\cite{uchikawa2003functional}}. In fact, these conserved non-coding elements have been found across independent vertebrate organisms. This is less so when comparing vertebrate with invertebrate species. 
	
%	Previously, the remainder of the non-coding regions were thought to be \emph{junk} DNA. This thinking has progressively changed over time and non-coding regions have been shown to be locations for transcription factor binding sites for the assistance in the regulation of gene expression. 
	
	Protein coding genes actually only make up a small portion of the genome. Estimates show that only approximately 2\% of the genome is protein coding, but a much larger 80\% is functional \textbf{CITE}. Enhancers are a subset of these non-coding regions that act as binding sites for proteins such as transcription factors and play a role in the regulation of protein synthesis.
	
	There are some competing models on the mechanisms behind enhancers. In short, the three main models are the enhanceosome, the billboard model and the TF collective; each of these suggest different mechanics and dynamics between TF-DNA and TF-TF pairings as well as the modes of activation for the enhancer. For a detailed review see Spitz and Furlong ~\cite{spitz2012transcription}. These enhancer regions are typically found flanking coding regions of DNA. Previous work by Algama~\cite{algama2017genome} have used statistical methods detailed in later sections to identify these non-coding functional elements.
	
% 	\subsection{Spitz and Furlong}
% 	{\color{red} Rework this section to fit in with Enhancers}
% %	Let us preface this section with this idea that nothing is ever definite in biology and that in general, it depends. 
% 	Enhancer regions are part of the subset of DNA that is considered to be non-coding. 
% 	Some of these enhancer regions can be sites for transcription factor binding and are thought to play important roles in the process of gene expression. 
% 	There two competing models proposed about how these enhancer regions work in the recruitment and attraction of transcription factors and how they encourage binding. 
% 	That being said, the general consensus seems to be that there is no global rule that determines the nature of the binding of transcription factors to these regions. 
% 	 {\color{red} TALK ABOUT DIFFERENT MODELS.}
	 
% 	 {\color{blue}Also, there have been some hypothesis about the importance of these regulatory modules in the developmental phase of organisms. It is thought that the programmed regulation of certain genes might be the cause for phenotypical differences among species. One notable example of the is the presence of the same family of genes being present across organisms spanned by a large evolutionary distance in the cardiac genes leading to differences in the amount of heart chambers that are present. }{\color{red}There's no citation. MIRANAS STUDY?}
	
	
% 	{\color{green}What is even cooler and might add evidence towards the fact that these regulatory regions are so important is that these non-coding elements in the DNA are very slow evolutionarily. So, like, that means they must be really important for all organisms, right? Another thing that's really  cool is that there are these regions around genes called gene deserts and the presence of certain proteins around these regions drops off really dramatically. Now, this might mean that there are inhibitors and stuff around the place that make sure that only the gene relevant to the proteins is being properly maintained.  }{\color{red} this is sort of up there}
    
\section{Sequence Alignment}
    Aligning of sequences can be posed as a mathematical problem. Consider two sequences $S_1$ and $S_2$ (possibly of different lengths) with residues from some finite alphabet $D$. Let one of the sequences (usually the longer one) be the \textit{reference sequence}; this is the sequence that the other one will be aligned to. The typical alignment approach considers the edit distance between the two sequences. This is achieved by either inserting, deleting or substituting characters in certain positions in the sequence to be aligned and assigning penalties to these operations, while scoring matching positions. An alignment is obtained through using a scoring scheme and optimizing an objective function. 
{\color{red}Disclaimer: This might not actually be how it works.}
A standard implementation of pairwise sequence alignment is obtained by considering the edit distance between two sequences and optimizing a scoring function. The edit distance is obtained by considering the number of insertions, deletions and substitutions of residues in the sequence to be aligned relative to the reference sequence. To achieve an optimal pairwise alignment the two sequences are stored in a table with the residues in columns and a (possibly negative) score is assigned to each column. Typically, columns where insertions, deletions or substitutions have taken place incur a penalty, while matching sequence entries have an associated positive score. The scoring system plays a large role in the generation of the alignment. As such, at least in the field of bioinformatics there are widely accepted scoring schemes such as PAM or BLOSSUM~\cite{mount2008comparison}. The selection of these scoring matrices typically depends of the application.

A variety of algorithms exist for the purpose of multiple sequence alignment (MSA) as opposed to just a pair of them. The class of algorithms associated with alignment problems, lends itself to the style of dynamic programming. Finding a pairwise alignment of two sequences is typically obtained by inserting gaps into the shorter sequence that maximizes the similarities of the input sequences according to some scoring matrix~\cite{edgar2006multiple}. One of the standard approaches for constructing multiple sequence alignments for $N$ sequences is to do $N-1$ pairwise alignments of pairs of sequences, under guidance from phylogenetic trees~\cite{feng1987progressive}.
	The alignment of nucleotide or protein sequences is useful because it can give insight into the different mutations that have occurred through genetics that give rise to different phenotypes. 
	\subsection{Genetic Segmentation}
	The act of segmenting a genetic sequence according to some criterion is known as genetic segmentation. There is some debate as to whether or not the genome actually follows a segmented structure {\color{red}CITE}. 
	
PAM OR BLOSUM \cite{mount2008comparison}
	
\section{Statistical Methods}
\subsection{Statistical Inference}
Consider some data $D$ where the observations $X_i$ composed of individual observations $X_i\sim f(x|\theta)$. A classical problem in statistics is the estimation of the parameter $\theta$. The typical, frequentist approach to this problem is to obtain point estimates of $\theta$ for example, by maximum likelihood estimation. The Bayesian methodology provides an alternative approach to this problem. In this framework, the parameter $\theta$ is treated as a random variable one seeks to infer the distribution of. Given some prior distribution (belief) about the parameter $p(\theta)$ and the likelihood of the data conditional on the parameters $p(D|\theta)$ the posterior distribution $ p(\theta|D)$ for the parameter conditional on the data can be obtained by using Bayes' rule
    \begin{equation}
        p(\theta|D) = \frac{p(D|\theta)p(\theta)}{p(D)}.
    \end{equation}
The term in the denominator is known as the marginal likelihood $p(D)$. In practice the marginal likelihood is often difficult or intractable to obtain so the rule is often written as 
    \begin{equation}
        p(\theta|D) \propto p(D|\theta)p(\theta)
    \end{equation}
Often, it is be very difficult to sample from the posterior distribution directly since there may be no closed form for the distribution except in the simplest toy problems. In practice, {\color{red} Markov chain Monte Carlo (MCMC) is deployed in order to sample from these distributions. \textbf{Need to stress MCMC powers}} 
% The advantage using a Bayesian framework is that rather than giving point estimates for quantities of interest, we are working in distributions and are able to give a have a measure of uncertainty. Given some prior distribution (belief) about our parameter $p(\theta)$ and the likelihood of the data, given the parameters $p(D|\theta)$ we can obtain what is known as the posterior distribution $ p(\theta|D)$ of the parameters given the data 
%     \begin{equation}
%         p(\theta|D) = \frac{p(D|\theta)p(\theta)}{p(D)}
%     \end{equation}


\subsection{Sequence Segmentation}
Change detection is a statistical problem concerned with determining when a change has occurred in a stochastic process underlying data generation. 
Consider a game with $k_{max}$ $D$-sided dice, each with a different probability distribution governed by a parameter $\theta$ such that $\theta_i \neq \theta_j$. One of the dice is rolled an unknown number of times and changed for another. This process is repeated $\kappa \leq k_{max}$ number of times. A record of each of the outcomes of all dice are recorded. Given only the concatenation of all the outcomes, the goal of change point detection is to determine when the dice were changed and the distribution of each die. 
% Consider a game where there are two coins with probability $\theta_1$ and $\theta_2$ of producing heads, with $\theta_1 \neq \theta_2$. One of the coins is flipped and unknown number of times and exchanged for the other, which is also flipped an unknown number of times. 
% The sequence of the flips is recorded in sequence and this data is all that is known, as well as the number of coins.
% In this example, the goal of change detection is to determine at what point the coins were exchanged as well as the estimation for the parameters $\theta_1$ and $\theta_2$. 

In bioinformatics, it common for the number of change points to be unknown {\color{red}perhaps corresponding to the start and end of putative functional elements}; this is a typical extension of change point detection. To generalize the above problem to this context, the data of primary concern are sequences of amino acids or nucleotide bases with an unknown number of segments. 
Instead of only dealing with two dice, there may be $k_{max}$ $D$-sided dice (with outcomes $\{1,\ldots,D\}$) having different biases (i.e that $\theta_i\neq \theta_j$ for $i\neq j$). Each die is rolled $C_i$ times in sequence until they have used $\kappa\leq k_{max}$ of the dice a total of 
    \begin{equation}
        J = \sum_{i=1}^{\kappa}C_i
    \end{equation}
number of times. Given only the concatenation of the outcomes $S$, the goal of the change point problem is to be able to determine the points where the opponent has switched dice.
{\color{red}SOMETHING ABOUT SLIDING WINDOW. One approach to a change dection problem is the use of a sliding window analysis that take a sldkfasldkfjas;dlfkj something something. It is sensitive to window size. Noise. etc}

The following is an outline of a Bayesian method for detecting change points, originally presented by Liu and Lawrence \cite{liu1999bayesian}. We define the change points $A_k$ to occur the first time a new die is used. That is
    \begin{equation}
        A_k = \sum_{\nu=0}^{k}C_\nu + 1
    \end{equation}
where $C_\nu$ is the number of times the $\nu$th dice was rolled, $C_0 = 0$ and $k = 1,\ldots,\kappa$. The quantities of interest are the number of change points $\kappa$ and their locations $\*A = (A_1,\ldots,A_\kappa)$. {\color{red} Work on a general framework on how sequence segmentation wotks in Bayesian}

\textbf{In the log file:}
    \begin{itemize}
        \item Number of changepoints
        \item Mixture values $\pi$ ($T$ many values)
        \item $\beta$ parameters ($T$ groups of $D$ many values)
    \end{itemize}
    
where $T$ is the number of segment classes/groups and $D$ is the size of the character alphabet.

\section{Markov Chain}
% \section{Definitions}
%     \begin{itemize}
%         \item A sequence$X_1, X_2, ...$ or random elements of some set is a \textbf{Markov Chain} if 
%             $$P(X_{n+1}|X_n,...,X_1) = P(X_{n+1}|X_n) $$
%         \item The transition probabilities are called \textbf{stationary} is $P(X_{n+1}|X_n)$ does not depend on $n$.
        
%         \item The marginal distribution of $X_1$ is called the \textbf{initial distribution}.
        
%         \item The distribution of $X_{n+1}|X_n$ is called the \textbf{transitional probability distribution}.
        
%         \item It's probably safe to assume stationary transition probabilities.
        
%         \item A Markov Chain is stationary iff the marginal distribution of $X_n$ does not depend on $n$.
        
%         \item An initial distribution is said to be \textbf{stationary} or \textbf{invariant} or        \textbf{equilibrium} for some transition probability distribution if the Markov Chain       specified by the initial distribution and transition probability distribution preserves the initial distribution. \textbf{Stationarity implies stationary transition probabilities but not vice versa.}
        
%         \item A transition probability distribution is \textbf{reversible} with respect to an initial distribution is, for some Markov Chain $X_1,X_2,...$ they specify, the distribution of pairs $(X_i,X_{i+1})$ is exchangeable.
        
%         \item A Markov Chain is reversible if its transition probability distribution is reversible wrt to its initial distribution. \textbf{Reversibility implies stationarity, but not vice versa.}
        
%         \item A Markov Chain is said to be \textbf{irreducible} if it is possible to get to any state from any state.
        
%         \item A state $i$ has period $k$ if any return to state $i$ must occur in multiples of $k$ time steps. 
%             $$k = gcd\{n>0:P(X_n = i|X_o = i) > 0\}$$
%         If $k=1$, then the state is said to be aperiodic. A Markov chain is \textbf{aperiodic} if all states are aperiodic. 
        
%         \item Define 
%             $$f_i = P(X \text{ ever returns to }i|X_0 = i)$$
%         State $i$ is said to be recurrent if $f_i = 1$ (or transient if $f_i < 1$.) I imagine that a Markov Chain is recurrent if all the states are recurrent?
        
%         \item For the \textbf{Ergodic Theorem} to apply, we require that a Markov Chain. An irreducible Markov Chain only needs one state to imply all states are aperiodic.
        
        
%     \end{itemize}
    
A Markov chain is a sequence $X_{t_1}, X_{t_2}, ...$ of random elements in some set such that the distribution of transition probabilities at the next state of the chain, conditional on all previous states, depends only on the current state of the chain; that is 
    \begin{equation}
        P(X_{t_{n+1}}|X_{t_n},X_{t_{n-1}},\ldots X_{t_1}) = P(X_{t_{n+1}}|X_{t_n}).
    \end{equation}
% {\color{red}
% {I'm not sure if the states should be upper case or lower case}}. 
For illustrative purposes, here we only consider discrete time Markov chains on discrete state spaces. 
The purpose of Markov chain Monte Carlo (MCMC) is generating a Markov chain so that the limiting distribution of $X_t$ as $t\rightarrow\infty$ is some $\pi$. 
In practice, this $\pi$ is the posterior distribution obtained for some quantities of interest in a Bayesian framework. 
Ensuring that the distribution of $X$ converges to $\pi$, the chain must be constructed with certain properties. However, to describe these mathematically it is necessary to define the probability that the chain moves from state $i$ to state $j$ in $t$ steps as $ P_{ij}(t) = P(X_t = j | X_0 = i)$
and the first return time to state $i$ to be $\tau_{ii} = \min\{t>0:X_t = i|X_0 = i\}$
    \begin{enumerate}[(i)]
        \item A Markov chain is called \textbf{irreducible} if $\forall~i,j, \exists~t > 0$ such that $P_{ij}(t) > 0$.
        % \item An irreducible chain is \textbf{recurrent} if $P(\tau_{ii} < \infty) = 1$ for some $i$.
        % \item An irreducible recurrent chain is called \textbf{positive recurrent} if $E[\tau_{ii}]<\infty$ for some $i$.
        \item An irreducible chain is \textbf{aperiodic} if for some $i$
            $$gcd\{n>0:P(X_n = i|X_0 = i) > 0\} =1. $$
    \end{enumerate}
The irreducibility condition intuitively means that all states are reachable from any starting state in a finite number of steps, while requiring aperiodicity means that the chain will not oscillate between a subset of states in a periodic manner. 
% Lastly, positive recurrence guarantees that once an iteration of the chain is sampled from the stationary distribution $\pi$, all subsequent iterations will also be distributed according to $\pi$.


\subsection{Notes}
    Try learn about markov chains in a countable vs general state space. See how theorems are posed and what differences they have in both cases. See how this then relates to the MCMC theorems about convergence. I THINK that Provided that a Markov chain, with a unique stationary distribution $\pi$ is irreducible and aperiodic, the chain will converge to $\pi$ i.e. $X_n \rightarrow \pi$ as $n\rightarrow\infty$

\section{Markov Chain Monte Carlo}
	The primary purpose of Markov chain Monte Carlo (MCMC) in the context of Bayesian inference is to sample from posterior distributions for which more efficient means of sampling are not available.
	%obtained in doing Bayesian analysis. 
	{\color{red}This is achieved by constructing a Markov Chain that is aperiodic and irreducible, (therefore ergodic) to ensure that the limiting distribution of the Markov chain is the the one that is desired to be sampled from. This method guarantees that $X_t\rightarrow\pi$ in in distribution $t\rightarrow\infty$}
% 	{\color{red}Apparently we only ever make Markov Chains that are reversible. I think this might be so that the detailed balance equations hold, which is sufficient condition for ergodicity(?)}
% 	\textbf{Practical considerations:} Mixing, convergence, proposal distributions, how long to run the chain. 
	\subsection{Metropolis-Hastings Algorithm}
	The Metropolis-Hastings (MH) Algorithm is the standard introduction to MCMC methods for sampling. Presented in 1970 by Hastings \cite{hastings1970monte}, it is an extension of the algorithm originally put forth by Metropolis in 1953 \cite{metropolis1953equation}.
	The steps for the algorithm, also presented in Algorithm~\ref{alg:MetHast}, are as follows. Let $\pi(x)$ be the target density (up to a constant), defined on some set ${\mathscr X}$.
	To initialize the algorithm let $x_0$ to be the first sample, chosen arbitrarily. In each iteration $g(y|x)$ 
% 	({\color{red}Is it more conventional to write $g(Y|X)$ or does it not matter as long as it is consistent?}), 
    known as the {\color{red}transition kernel} is sampled from to generate a candidate point. The candidate point $y$ is accepted with probability 
	    \begin{equation}
	        \alpha(x,y) = min\left(1,\frac{\pi(y)g(x|y)}{\pi(x)g(y|x)}\right).
	    \end{equation}
	If the candidate point is accepted, $x_{t+1} = y$, otherwise, $x_{t+1} = x_t$ and the next iteration is begun. In the original Metropolis algorithm, $g$ is chosen to be symmetric; that is $g(y|x) = g(x|y)$. However, in general $g$ can be arbitrary. There are smart ways to tune $g$ to ensure that the chain mixes adequately.
	
	\begin{algorithm}[H]
	\label{alg:MetHast}
 %\KwData{this text}
 %\KwResult{how to write algorithm with \LaTeX2e }
 Set $x_0$, $t=0$\; 
 \Repeat{converged}{
  Draw $y\sim g(y|x)$\;
  Draw $u\sim U(0,1)$\;
  \eIf{$u<\alpha(x,y)$}{
   $x_{t+1} = y$\;
   }{
    $x_{t+1} = x_t$\;
  }
  Increment $t$\;
 }
% 		\begin{enumerate}
% 			\item Generate a candidate point $x'$ from $g(x'|x_0)$.
% 			\item Calculate the acceptance probability
% 				\begin{equation}
% 				\alpha = \frac{f(x')}{f(x_t)}
% 				\end{equation}
% 				and generate $r\sim U(0,1)$ and compare with $\alpha$ to decide whether or not to accept the candidate point probabilistically.
% 			\item Repeat steps 1. and 2.
% 		\end{enumerate}
% 	\subsection{Gibbs Sampler}
% 	The Gibbs sampler is presented as a method for sampling from some distribution over a set ${\mathscr X}$ with dimension $n$. Let $x = (x_1,...,x_n)$ and let us denote the $i^{th}$ sample as $x^{(i)} = (x_1^{(i)},...,x_n^{(i)})$.


 \caption{Metropolis-Hastings Algorithm}
\end{algorithm}

\subsection{Gibbs Sampler}
In general, the quantity $x$ is a vector and standard MH updates all the components at once. In contrast to this, there exist a class of MCMC methods known as Gibbs samplers, that iteratively update the components of $x$. The Gibbs Sampler originally presented by  Geman and Geman \cite{geman1984stochastic} is an example of one of these and can be shown to be a special case of single component MH \cite{gilks1995markov}. 
% Rather than updating the entirety of $x$ as in the MH Algorithm, there exist a class of MCMC algorithms that update components of $x$ in an iterative fashion, the most popular of these being the Gibbs Sampler, originally formulated by Geman and Geman \cite{geman1984stochastic}. 
Denote $x_. = (x_{.,1},\ldots x_{.,h})$. Note that each of the components of $x$ may possibly be of differing dimension. 
% ({\color{red} I suspect that this has something to do with efficiency. I.e. you might want to group components together that have 'nice' conditional distributions?}).
Introducing the notation $x_{.,-i} = (x_{.,-i},\ldots x_{.,i-1}, x_{.,i+1} \ldots x_{.,h})$, that is $x_{.,-i}$ comprises of all the components of $x_.$ except $x_{.,i}$. Gibbs sampling updates $x_{.,i}$ by drawing a candidate $y_i$ from $\pi(y_{.,i}|x_{.,-i})$, also known as the full conditional distribution. An advantage of the Gibbs Sampler is that all candidate points are accepted with probability 1.

\begin{algorithm}[H]
 %\KwData{this text}
 %\KwResult{how to write algorithm with \LaTeX2e }
 Set $x_0$, $t=0$\; 
 \Repeat{converged}{
    \For{$i$ = 1 to $h$}{
  Draw $y\sim \pi(y_{t,i}|x_{t,-i})$\;
  Set $x_{t,i} = y$
  }
  Set $x_{t+1} = x_t$\;
  Increment $t$\;
 }
  \caption{Gibbs Sampling Algorithm}
\end{algorithm}

\section{When does MCMC have problems?}
Traditional MCMC algorithms like (MH or Gibbs will work nicely in the majority of applications when the number of parameters wanting to be inferred is fixed and hence the target distribution is defined on a space of fixed dimension. In general, once the model is established, the application of MCMC for posterior inference is straight forward. In some situations, the posterior may be sensitive to model selection; for example, when the number of unknowns wanting to be inferred is also unknown. This leads to a situation where the posterior is defined on a space of unknown or unfixed dimension. A mention is made to Green who had originally proposed a MCMC method known as Reversible Jump Markov Chain Monte Carlo in response to this model selection problem by including the number of parameters into the inference  (for technical details see \cite{green1995reversible,  waagepetersen2001tutorial}). Instead, the generalized Gibbs sampler is presented below as an alternative way to sample from these spaces.
\section{Generalized Gibbs Sampler}
%So, here is my attempt to explain in my own words how the Generalised Gibbs Sampler (GGS) works!

\subsection{Preamble} 
Samplers based on MH or Gibbs can work very well when the distributions of interest are defined on spaces of fixed dimension. In the context of gene analysis, sometimes the distributions that need to be sampled from are defined on spaces without a fixed dimension. Standard MH or Gibbs fail in spaces like this but there exists a method known as the Generalised Gibbs Sampler, presented by Keith \emph{et al}. \cite{keith2004generalized}. {\color{red}{It can be shown that reversible jump is a speciial case (MAYBE MENTION THIS AT END?)}} 
Rather than creating a Markov Chain on ${\mathscr X}$ this algorithm is generates a Markov Chain in the set $\mathscr U \subseteq \mathscr I \times \mathscr X$. Where $\mathscr I$ denotes the index set; a set that keeps indices of the type of transitions available at each step of the chain. The set $\mathscr U$ must be chosen such that the projections of $\mathscr U$ on the sets $\mathscr I$ and $\mathscr X$ are onto. The algorithm functions in two steps: the \textbf{Q Step} and the \textbf{R Step}, detailed below. 

\subsection{Q Step} The Q step is about deciding what type of transition to make next. 
To describe the Q step, it is necessary to define some objects. Let $\*u = (i,x) \in {\mathscr U}$, wehere $i\in \mathscr I$ and $x\in\mathscr X$.
The set $\mathscr Q(x) = \{(i,z) \in \mathscr U : z = x\}$ is defined as the set of types of possible transitions available at $x$.
Also, for every $x \in \mathscr X$ a transition matrix $\mathcal Q_x$ is defined on $\mathscr Q(x)$. Let $q_x$ be a distribution stationary with resepct to $\mathcal Q_x$, required for a subsequent step. 
% {\color{red}I think this means that we have to construct $\mathcal Q_x$ to be aperiodic to ensure the existence of $q_x$. Is the stochastic process represented by $Q$ also a Markovian? Probably.}
Also defined is
	\begin{equation}
		Q(\*u,\*v) = 
			\begin{cases}
			\mathcal Q_x(\*u,\*v), & \text{for } \*v \in \mathscr Q(x)\\
			0 & \text{otherwise} 
			\end{cases}
		\end{equation}
on $\mathscr U$ to be a global transition matrix from $\*u = (i,x)$ to $\*v = (j,y)$.

Given that the system is in state $\*u = (i,x)$, carrying out the Q step is done by generating $\*v \in \mathscr Q(x)$, with a draw from the distribution of with density $Q(x,\cdot)$.

\subsection{The R Step}
Once the Q step has been performed, the R step is then about generating the next point in the chain. Again, it is necessary for some definitions.
For each $\*u \in \mathscr U$, define the set $\mathscr R(\*u)$ as the set of possible transitions from $\*u$. The sets $\mathscr R(\*u)$ must form a partition of ${\mathscr U}$. 
% {\color{red} Not entirely sure why this is a requirement. What happens if they don't form a partition? My initial guess is that there will be some states that aren't accessible OR YOU COULD JUST NOT HAVE THEM BE IN $\mathscr U$}
Define $$\mathcal R_\*u(\*v) = \frac{s(\*u,\*v)f(y)q_y(\*v)}{\sum_{\*w\in\mathscr R(\*u)}f(z)q_z(\*w)}$$ to be the probability to transition from state $\*u$ to state $\*v\in\mathscr R(\*u)$. Here, $s$ is some symmetric, non-negative function and $f$ is the distribution of interest {\color{red}{should I change this to $\pi$ or change the others to $f$?}}. 
A global transition matrix on ${\mathscr U}$ is defined to be
\begin{equation}
		R(\*u,\*v) = 
			\begin{cases}
			\dfrac{s(\*u,\*v)f(y)q_y(\*v)}{\sum_{\*w\in\mathscr R(\*u)}f(z)q_z(\*w)}, & \text{if } \*v \in \mathscr R(\*u)\setminus\{\*u\}\\
			1 - \sum_{\*w \in \mathscr R(\*u)\setminus\*\{u\}} R(\*u,\*w) & 
		    \text{if $\*v = \*u$}\\
			0 & \text{otherwise}\end{cases}
		\end{equation}
	where $\*w = (k,z)$ and $\sum_{\*w \in \mathscr R(\*u)\setminus\*\{u\}} R(\*u,\*w) \leq 1$.
After having selected $\*v \in \mathscr Q(\*u)$ by performing the Q step, generate $\*w \in \mathscr R(\*v)$ by drawing from $R(\*u,\cdot)$ and update the state of the chain to $\*w$.
\subsection{Algorithm (GGS)}
Repeating the Q step and R step of the algorithm generates a Markov chain $\{\*u_0,\*u_1,\ldots\}$ on $\mathscr U$ such that the projection on to $\mathscr X$ is the target distribution $\pi$.

\begin{algorithm}[H]
 %\KwData{this text}
 %\KwResult{how to write algorithm with \LaTeX2e }
 Set $\*u_0$, $t=0$\; 
 \Repeat{converged
 }{
    Generate $\*v\in\mathscr Q(x) \sim \mathcal Q(\*u,\cdot)$\;
    Generate $\*w\in\mathscr R(\*v)\sim R(\*v,\cdot)$\;
    Set $\*u_{t+1} = \*w$\;
    Increment $t$\;
 }
  \caption{Algorithm for the generalized Gibbs Sampler}
\end{algorithm}


% While the Gibbs sampler is used to draw from $n$ dimensional distributions, the Generalised Gibbs sampler, presented by Keith et al.\cite{keith2004generalized} can be used to draw from distributions of varying dimension. Suppose that we wish to sample from $f$ on a space $\mathscr X$. To outline the process that GGS uses to generate a Markov chain, consider a set ${\mathscr I}$ called the index set. It serves the purpose of cataloguing the types of transitions that are allowed to happen. Let ${\mathscr U = \mathscr I\times \mathscr X }$. $\mathscr U$ must be chosen such that the projections of $\mathscr U$ on the sets $\mathscr I$ and $\mathscr X$ are onto. The GGS operates in two steps: the $Q$ step, where the choice of transition types is made and the $R$ step where a transition of that type is made. To do the $Q$ step define $\mathscr Q(x)$ to be the set $\mathbf{u}=(i,x)\in \mathscr U$ for a given $x$. This set acts as a catalogue for the type of transitions that may be taken from $x$. Let $Q_x$ on $\mathscr Q(x)$ be the transition matrix used as a means for selecting the type of transition we may take from some $x\in\mathscr X$. The choice of $Q_x$ is arbitrary except for the condition that it must be irreducible to guarantee the existence of the stationary distribution $q_x$. If we denote $Q_x((i,x),(j,y))$ to be the probability of transitioning selecting transition type $j$. We let
% 		\begin{equation}
% 		Q(\*u,\*v) = 
% 			\begin{cases}
% 			Q_x(\*u,\*v), & \text{for } \*v \in \mathscr Q(x)\\
% 			0 & \text{otherwise} 
% 			\end{cases}
% 		\end{equation}
% 	denote choosing this transition type for $\*u=(i,x)$ and $\*v = (j,y)$. 
% 	Next, to perform the $R$ step, define $\mathscr R(\*u)$ as the set of possible transitions from $\*u$. These sets must be chosen such that they form a partition of $\mathscr U$. The transition from $\*u$ to some other state $\*v = (j,y)$ is governed by the matrix
% 		\begin{equation}
% 		R(\*u,\*v) = 
% 			\begin{cases}
% 			\dfrac{f(y)q_y(\*v)}{\sum_{\*w\in\mathscr R(\*u)}f(z)q_z(\*w)}, & \text{for } \*v \in \mathscr R(\*u)\\
% 			0 & \text{otherwise}\end{cases}
% 		\end{equation}
% 	where $\*w = (k,z)$. To summarise, the steps in the algorithm are as follows
% 		\begin{enumerate}
% 			\item Initialize state $U_i$.
% 			\item Select the type of transition, by using the matrix $Q_x(\*u,\*v)$. (Q Step)
% 			\item Perform a transition of the selected type by using $R(\*u,\*v)$. (R Step)
% 			\item Set the result of this transition as $U_{i+1}$.
% 			\item Repeat steps 2-4.
% 		\end{enumerate}
% 	The generates a Markov Chain in $\mathscr U$ where the projection on $\mathscr X$ is the distribution $f$.

One of the benefits of the GGS is that it can sample from non-standard spaces where conventional Gibbs sampling may fail. Because its general nature the earlier MCMC samplers, including the Reversible Jump MCMC can be shown to be special cases of the GGS, with appropriate choice of Q,R and $s$.  %{\color{red}{This algorithm has been applied to sequence segmentation and stuff}}

\subsection{Practical Considerations}

% \section{Reversible Jump MCMC}
% The Reversible Jump Markov chain Monte Carlo algorithm was originally presented as a way to select against competing candidate models with different numbers of parameters. It is also able to sample from sets where the elements may not be of fixed dimension. 

% \subsection{Preamble}
% Suppose you have a collection of models $\{\mathscr M_k : k\in\mathscr K\}$ such that model $\mathscr M_k$ has an associated parameter vector $\theta^{(k)} \in \mathbb{R}^{n_k}$. The set $\mathscr K$ is known as the index set and $n_k$ may differ in each model. Given observed data $D$, we can express the joint distribution of $(k,\theta^{(k)},D)$ as product of the model probability, prior and likelihood
%     \begin{equation*}
%         p(k,\theta^{(k)},D) = p(k)p(\theta^{(k)}|k)p(D|\theta^{(k)},k).
%     \end{equation*}
% For convenience, we denote $x = (k,\theta^{(k)}) \in \{k\}\times\mathscr
% C_k$ and in general, $x$ will vary over $\mathscr C = \cup_{k\in\mathscr K}\mathscr C_k$

% \subsection{Algorithm}
% Suppose the chain is in current state $x_t = (k,\theta^{(k)})$. To generate a candidate point $y_{t+1}= (y^{ind}_{t+1},y^{par}_{t+1})$, we generate a candidate for the model indicator $y^{ind}_{t+1} = k' \in \mathscr K$ and parameter vector $y^{par}_{t+1} = \theta^{(k')}\in \mathbb{R}^{n_{k'}}$. 

%     \begin{algorithm}[H]
%     Set $x_0$, $t=0$\; 
%     \Repeat{you're happy}{
%     Generate $y^{ind}\sim p(k)$\;
%     $y^{par}=g_{kk'}(x,u)$\;
%     Generate $r\sim U(0,1)$\;
%         \eIf{$r<\alpha_{kk'}(x,y)$}{
%         $x_{t+1} = (y^{ind},y^{par})$}{
%         $x_{t+1} = x_t$}
%     Increment $t$\;
%     }
%   \caption{Reversible Jump MCMC}
% \end{algorithm}
% \subsubsection{Special things}
% {\color{red}{For reasons I don't understand,}} it is convenient to generate the candidate $y^{par}_{t+1} = g_{1kk'}(x,u)$, where $u$ is some random vector on $\mathbb{R}^{n_{kk'}}$ with density $q_{kk'}(x,\cdot)$ and $g_{1kk'} : \mathbb{R}^{n_k + n_{kk'}}\rightarrow\mathbb{R}^{n_k'} $ is a deterministic function.
% \\ \\
% Consider the move from $(k,x)\to (k',x') = (k',g_{1kk'}(x,u))$ as well as the reverse move $(k',x')\to (k,x)=(k,g_{1k'k}(x',u'))$ in order for the current state of the Markov chain to have the same dimension as the proposed point, we must impose the dimension matching condition    
%     \begin{equation}
%         n_k + n_{kk'} = n_k' + n_{k'k}.
%     \end{equation}
% This ensures that the joint densities $f(x|m)q_{kk'}(x,u)$ and $f(x'|k')q_{k'k}(x',u')$
% are defined on spaces of equal dimension.
% \\ \\
% Furthermore(!) there are also assumptions that there the functions $g_{2kk'}:\mathbb{R}^{n_k + n_{kk'}} \to \mathbb{R}^{n_{k'k}}$ and $g_{2k'k}:\mathbb{R}^{n_{k'}+n_{k'k}}\to \mathbb{R}^{n_{kk'}}$ exist so that the mapping 
%     \begin{equation}
%         (x',u') = g_{kk'}(x,u) = (g_{1kk'}(x,u),g_{2kk'}(x,u))
%     \end{equation}
% is one-to-one with
%     \begin{equation}
%          (x,u) = g^{-1}_{k'k}(x',u') = (g_{1k'k}(x',u'),g_{2k'k}(x',u'))
%     \end{equation}
% and that $g_{kk'}$ is differentiable (Because we need the Jacobian of this in the acceptance probability.)
% \\ \\
% Candidate points are accepted with probability
%     \begin{equation}
%         \alpha_{kk'}(x,x')=min\left(1,
%         \frac{p(K = k')f(x'|K=k')p_{k'k}q_{k'k}(x',u')}{p(K=k)f(x|K=k)p_{kk'}q_{kk'}(x,u)}\left|\frac{\del g_{kk'}(x,u)}{\del x\del u}\right|\right) 
%     \end{equation}
% where
%     \begin{equation}
%         \frac{\del g_{kk'}(x,u)}{\del x\del u} = 
%             \begin{bmatrix}
%                  \frac{\del g_{1kk'}(x,u)}{\del x}       & \frac{\del g_{2k'k}(x,u)}{\del x}\\
%                  \frac{\del g_{1kk'}(x,u)}{\del z}       & \frac{\del g_{2k'k}(x,u)}{\del x}
%             \end{bmatrix}
%     \end{equation}
% \section{Comments}
% The RJMCMC is a special case of the GGs, but I'm still ironing out the details!
% \\ \\
% The most burning question I have is how does one know what their target distribution $\pi$ look like? My feeling is that, in practice, $\pi$ is typically the posterior distribution obtained in the Bayesian framework and expressed in terms of the prior and likelihood distributions.
% \\ \\ 
\newpage
\section{Enhancers}
    Transcription factors are important because they modify the transcriptional machinery
    
    How do they do it?
        Gene expression is crucial to the function and differentiation of different cell types and required for growth, repair and maintenance of an organism.  In eukaryotic cells, genes are embedded in DNA and collected in tightly packed configurations, known as chromatin, in the nucleus of cells~\cite{alberts2002chromosomal}.  A gene is a region in the DNA that contains the necessary information for the synthesis of an associated protein.
        The types of genes and the intensity associated with their expression in a cell is known as an expression profile. All the nucleated cells in an organism contain the same DNA, but the differences in their function arises from the different expression profiles~\cite{lockhart2000genomics}. Our interest lies in the identification and classification of certain regions in the DNA that help regulate the expression of particular genes. 
       

        One of the enduring tenets of molecular biology is that gene expression occurs as a result of DNA being transcribed to RNA and then translated to proteins~\cite{crick1958protein}.During the first stage, proteins called transcription factors (TFs) bind to regulatory regions surrounding the gene to recruit RNA polymerase; an enzyme responsible for the synthesis of a sub type of RNA known as messenger RNA (mRNA). 
        The second phase, known as translation, is primarily carried out by a molecule called a ribosome and by translational RNA (tRNA) outside of the nucleus. The end result is a protein comprised of an amino acid chain.
        
        Gene expression can be regulated in a variety of stages, but we focus on regulation in the context of transcription. 
At the transcriptional level, regulation requires involvement of a large number of factors, proteins and enzymes with a variety
 of roles~\cite{lemon2000orchestrated}. 
The main mechanism by which TFs interact with DNA, and subsequently regulate transcription
. the In this work, we focus on transcriptional regulation, through enhancers.

TFs are able to recognize short sequences in non-coding regions of DNA and bind to these sites. 
Upon binding, TFs recruit other TFs to bind to other sites on the DNA or to the proteins themselves.
The exact mechanism of how these TF complexes work is not understood very well. 
One plausible explanation is that some combination of TFs need to be succesfuly bound to DNA to carry out their function.
Another theory suggests that TFs can bind in a nonb-specific way to DNA sequences as well as other TFs, introducing 
a dimension of protein-protein interaction in to the activation of the reglatory element. 
There are other suggestions on how TFs might bind cooperatively in order to become active regulatory elements; for a review, 
see \textbf{Spitz and Furlong 2012 CITE}. 


Here are the main ways that gene expression might be altered:
\begin{itemize}
	\item Direct changes to DNA sequence. This is known as mutation and is typically "observed" as an organism evolves
		For example, we can see in relatively closely related species D.Melangocaster and D.Simulans, that the
		gene EXAMPLE GENE has mutated resulting in either a different expression pattern or a different gene.
		In extreme cases, this mutation may result in a different gene or even a silenced gene that is no longer
		expressed.
	\item Epigenetic changes. Gene expression is a complex and multifaceted process. Regulation in the expression
		of a gene happens at different levels, at different stages, by different factors. For the purposes
		of this work, a focus will be placed on regulation at the transcriptional level. Specifically, the focuse
		will be placed on a subset of TFs. We would like to identify and classify these enhancers at a stage
		at a sequence level.
\end{itemize}
        
TF binding sites (TFBS) are divided into categories according to the function associated with the TFs that bind to that site.
We would like to focus on the category known as Enhancers. Enhancers are a type of TFBS that enhance or upregulate the 
expression of a gene associated with that site.
To compount things a single enhancer may act on multiple genes at some distance away from the gene of interest. 


        Changes in gene expression can often be attributed to changes in either the direct genetic sequence (\emph{eg.} mutation), or epigenetic factors in the expression process. There are numerous ways that epigenetic factors can alter the expression of genes without changes to the genetic code itself. A readily available example of an epigenetic factor are TFs. These TFs take on a diverse set of roles in regulating transcription, such as the initiation of the process and the recruitment of other factors~\cite{lemon2000orchestrated}.
        
        
        Conservation of sequence as a single metric has been used for the identification of enhancers. This approach has proven useful in the past, but in the context of cardiac enhancers, relatively few have been found. Sequencing has been used, in conjunction with knowledge about binding proteins~\cite{blow2010chip}
        
        Enhancers may show conservation of function across a large evolutionary distance, even if the sequences have diverged significantly~\cite{tautz2000evolution}
        
        That doesn't mean that there is no loss of sequence conservation. It has been shown in tissue specific studies that gene expression is much higher conserved than TF binding. Positive correlation with combinatorial TF binding and transctipion levels of closeby genes~\cite{wong2014decoupling}
        
        We have some data in the form of ChIP-Seq h3K4me1. This denotes the methylation status of a histone. This particular site often associated with promoter regions of actively being transcribed genes~\cite{barski2007high}. 
        %https://epigenie.com/key-epigenetic-players/histone-proteins-and-modifications/histone-h3k4/
        While this association has been used to identify active and primed enhancers, it is poorly understood whether H3K4me1 influences, is influenced by or correlates enhancer activity~\cite{rada2018h3k4me1}
        
        The many faces of histone lysine methylation 2002)
        
        Revisit Spitz and Furlong to see what sort of idease we have to give an overlay of what a TF is and how we have different models for regulation. 
        
       
        
        
        
        As an example, the remodelling of chromatin through histone methylation can alter the accessibility of transcriptional machinery to reach their target genes~\cite{gibney2010epigenetics, holoch2015rna }. 
        
        % Changes in gene expression is primarily driven by direct changes in the sequence (genetic) or changes in epigenetic factors involved in the expression process, such as chrmomatin remodelling through histone methylation altering accessibility between transcriptional machinery and their target gene~\cite{gibney2010epigenetics, holoch2015rna }.% (RNA CAN ALSO REGULATE GENE EXPRESSION MAYBE) 
        Epigenetics is important~\cite{holliday2006epigenetics}
        DNA Methylation is super important when considering stability of gene epxression~\cite{jaenisch2003epigenetic}
        
        Comparative genomics might not be the best way since these things display modest or no conservation. Sequence conservation does not provide clues to function, even if it is identified as an enhancer~\cite{pennacchio2013enhancers}
        
        
        We have many different 

%\pagenumbering{gobble}
\bibliographystyle{acm}
\bibliography{DominicPhDLitReview}
\end{document}

% 	This section aims to outline the main ideas of the techniques and algorithms used in the analysis that are performed as well as including some indication to where they are applied.
% 	\subsection{Metropolis Algorithm}
% 	The Metropolis Algorithm is used to sample from distributions that might typically be intractable to sample from.
% 	The steps for the algorithm is as follows. Let $f(x)$ be some function that is proportional to $P(x)$ the distribution to be sampled from. To initialize the algorithm first choose $x_0$ to be the first sample. The distribution $g(y|x)$, known as the jump distribution, is sampled from to generate candidate points. In general, $g$ can be arbitrary with the restriction that it is symmetric. That is, $g(y|x) = g(x|y)$. 
% 		\begin{enumerate}
% 			\item Generate a candidate point $x'$ from $g(x'|x_0)$.
% 			\item Calculate the acceptance probability
% 				\begin{equation}
% 				\alpha = \frac{f(x')}{f(x_t)}
% 				\end{equation}
% 				and generate $r\sim U(0,1)$ and compare with $\alpha$ to decide whether or not to accept the candidate point probabilistically.
% 			\item Repeat steps 1. and 2.
% 		\end{enumerate}
% 	\subsection{Gibbs Sampler}
% 	The Gibbs sampler is presented as a method for sampling from some distribution over a set ${\mathscr X}$ with dimension $n$. Let $x = (x_1,...,x_n)$ and let us denote the $i^{th}$ sample as $x^{(i)} = (x_1^{(i)},...,x_n^{(i)})$.
% 		\begin{enumerate}
% 			\item Choose initial sample $x^{(i)} = (x_1^{(i)},...,x_n^{(i)})$.
% 			\item To generate $x^{(i+1)}$, each component is conditionally sampled from by iteratively updating each of the other components. That is, to sample $x_j^{(i+1)}$ from the full conditional distribution
% 				\begin{equation}
% 				p(x_j^{(i+1)}|x_1^{(i+1)},...,x_{j-1}^{(i+1)},x_{j+1}^{(i)},...,x_n^{(i)}).
% 				\end{equation}
% 			\item Set $x^{(i+1)}$ to be the sample obtained after $x_n^{(i+1)}$ has been updated.
% 			\item Repeat steps 2 and 3 above. 
% 		\end{enumerate}
% 	\subsection{Generalised Gibbs Sampler}
% 	While the Gibbs sampler is used to draw from $n$ dimensional distributions, the Generalised Gibbs sampler, presented by Keith et al.\cite{keith2004generalized} can be used to draw from distributions of varying dimension. Suppose that we wish to sample from $f$ on a space $\mathscr X$. To outline the process that GGS uses to generate a Markov chain, consider a set ${\mathscr I}$ called the index set. It serves the purpose of cataloguing the types of transitions that are allowed to happen. Let ${\mathscr U = \mathscr I\times \mathscr X }$. $\mathscr U$ must be chosen such that the projections of $\mathscr U$ on the sets $\mathscr I$ and $\mathscr X$ are onto. The GGS operates in two steps: the $Q$ step, where the choice of transition types is made and the $R$ step where a transition of that type is made. To do the $Q$ step define $\mathscr Q(x)$ to be the set $\mathbf{u}=(i,x)\in \mathscr U$ for a given $x$. This set acts as a catalogue for the type of transitions that may be taken from $x$. Let $Q_x$ on $\mathscr Q(x)$ be the transition matrix used as a means for selecting the type of transition we may take from some $x\in\mathscr X$. The choice of $Q_x$ is arbitrary except for the condition that it must be irreducible to guarantee the existence of the stationary distribution $q_x$. If we denote $Q_x((i,x),(j,y))$ to be the probability of transitioning selecting transition type $j$. We let
% 		\begin{equation}
% 		Q(\*u,\*v) = 
% 			\begin{cases}
% 			Q_x(\*u,\*v), & \text{for } \*v \in \mathscr Q(x)\\
% 			0 & \text{otherwise} 
% 			\end{cases}
% 		\end{equation}
% 	denote choosing this transition type for $\*u=(i,x)$ and $\*v = (j,y)$. 
% 	Next, to perform the $R$ step, define $\mathscr R(\*u)$ as the set of possible transitions from $\*u$. These sets must be chosen such that they form a partition of $\mathscr U$. The transition from $\*u$ to some other state $\*v = (j,y)$ is governed by the matrix
% 		\begin{equation}
% 		R(\*u,\*v) = 
% 			\begin{cases}
% 			\dfrac{f(y)q_y(\*v)}{\sum_{\*w\in\mathscr R(\*u)}f(z)q_z(\*w)}, & \text{for } \*v \in \mathscr R(\*u)\\
% 			0 & \text{otherwise}\end{cases}
% 		\end{equation}
% 	where $\*w = (k,z)$. To summarise, the steps in the algorithm are as follows
% 		\begin{enumerate}
% 			\item Initialize state $U_i$.
% 			\item Select the type of transition, by using the matrix $Q_x(\*u,\*v)$. (Q Step)
% 			\item Perform a transition of the selected type by using $R(\*u,\*v)$. (R Step)
% 			\item Set the result of this transition as $U_{i+1}$.
% 			\item Repeat steps 2-4.
% 		\end{enumerate}
% 	The generates a Markov Chain in $\mathscr U$ where the projection on $\mathscr X$ is the distribution $f$.
% \section{Genome Segmentation}
% Traditional methods for the segmentation of genomes has primarily relied on varying types of analysis such as sliding window or {\color{red}ETC.}. Some advancements in field of segmentation have come from the incorporation of Bayesian techniques, such as the generalised Gibbs sampler to segment eukaryote genomes \cite{keith2006segmenting}.
% The input required for the \texttt{changept} program is a sequence $S$ that is made from a finite alphabet $\{1,...,D\}$.
% A segmentation of $S$ is composed of the number of change points $k$ and a vector $A = (A_1,..., A_k)$ containing the positions of change points.
% Where $A_i$ is the position of the left most character in segment $i+1$.
% Every segment in the sequence is assumed to have been drawn from independent trials out of $D$ possible outcomes. 

% Let $\Theta_i = (\theta_{i,1},...,\theta_{i,D})$ be the vector containing the probabilities of the $D$ outcomes.
% The probability of an observed sequence is then a product of binomial distributions
% 	\begin{equation}
% 	p(S|k,A,\Theta) = \prod_{i = 1}^{k + 1}\prod_{j = 1}^{D}\theta_{i,j}^{m_{ij}}
% 	\end{equation}
% where $m_{ij}$ is the number of times that character $j$ appears in segment $i$. 

% A prior probability distribution must be specified. Let $\phi$ be the probability of a new a new segment. Then, 
% 	\begin{equation}
% 	p(k,A) = \phi^k(1-\phi)^{L-1-k}
% 	\end{equation}
% where $L$ is the length of $S$. The distribution
% 	\begin{equation}
% 	p(k,A,\Theta) = p(k,A)\prod_{i=1}^{k+1}p(\theta_i).
% 	\end{equation}
% Here, $p(\theta_i)$ is Dirichlet distributed with parameter vector $\alpha = (\alpha_1, ..., \alpha_D)$. The parameters $\phi$ and $\alpha$ are also incorporated into the inference. $\phi$ has prior distribution $Beta(a,b)$ and 
% 	\begin{equation}
% 	p(\alpha) \propto \left[\frac{\Gamma(\sum_{j}\alpha_j)}{\prod_j\Gamma(\alpha_j)}\right]^{c-1}e^{-d\sum_j\alpha_j}
% 	\end{equation}
